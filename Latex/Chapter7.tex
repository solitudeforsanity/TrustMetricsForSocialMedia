\chapter{Reflection}
\section{Conclusion}
It takes only a few seconds to post something on Twitter. It takes even less time, to retweet a news feed on Twitter. If thousands of people take to retweeting a piece of information news can go viral within minutes. The problem arises when a piece of information spread, is not trustworthy and is false information. This could harm a lot of people concerned, particularly if this is a crisis situation. Hence Trust Metrics is an important measure to have in social media, for a more reliable online experience.\\*\\*
In this dissertation, we considered one method to differentiate Trustworthy information from Untrustworthy information. We focused on the Twitter platform, and assumed that news tweets are Trustworthy, and learned characteristics from these news tweets and used this learning to predict if other tweets are Trustworthy or not. Likewise we learned the characteristics of Untrustworthy tweets from fake twitter profiles that post Untrustworthy information and similarly used this learning to predict if other tweets are Untrustworthy or not. \\*\\*
The aim of our project was to attain at least an 80\% accuracy rate, in predicting if tweets were Trustworthy or not. In order to achieve this we used three different Machine Learning Techniques, namely Logistic Regression, Support Vector Machines and Random Forests. Using these and a collected set of data of Trustworthy and Untrustworthy tweets, we trained and tested the model such that we gain the maximum accuracy rate as possible. We have shown the Random Forests and Logistic Regressions predict with an accuracy of over 85\%. \\*\\*
We can see from our results that we have have achieved our aim. We can conclude by saying that learning Trust features from News Tweets and Fake Twitter profiles is a reliable way of learning the different characteristics of Trustworthy and Untrustworthy tweets. 
\section{FutureWork}
Future work could include extending on the current model or taking a new approach to solving this problem and later comparing it with the current predictions we have. When it comes to extending the current model, I believe that there are quite a few features that we may be able to add to the current feature set. This may or may not improve accuracy or generalise the model but we could test with different features to check how the model performs. For instance as mentioned in the previous chapter we could perform more tweet content based analysis. We could analyse the twitter feeds in verifying any external links, videos or photos posted as well as words used, calibre of words used, the grammar punctuation etc.  Again as suggested earlier we could include third party analysis, i.e give a rating to someones followers, people they respond to, people they constantly tweet to or receive tweets from and use this to train the data as well. \\*\\*
I would also like to suggest a different approach to the problem of Trust Metrics. In this, we focus on current affairs or trending topics and gather all the tweets that are tweeted regarding the topic. (This could possibly be a hashtag or the trending hashtags). We also have a listed collection of News Websites from all over the world. In this method we check Trustworthiness purely based on the content of the message as opposed to other features. We would then check if the content that is posted on some Twitter users' profile is similar to or different to what the Trusted media would be tweeting about. For example if a statement like "The London bridge is on Fire" was posted by some person and we have another tweet by BBC mentioning that "The London bridge is NOT on fire" then we would classify the former tweet as Untrustworthy. It would be interesting to compare results we received using this method and the method we devised in this dissertation.