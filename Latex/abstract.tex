
\begin{abstract}

The proliferation of Internet usage has resulted in huge amounts of information being made available online. This information originates from private persons, businesses, governments and other such organisations and spans from news, opinions, personal information to rumours and unreliable information. The fact that there is a vast amount of information, sometimes with rumours or inconsistencies makes it increasingly difficult to know who or which information source to trust.\\*\\* 
Reputable sources such as news agencies would be ideal to trust, however we also instinctively trust our closest friends or relatives, or information presented to us by people in our social or work networks. One significant challenge therefore is in cases where these known trustworthy sources do not have knowledge of a situation or do not possess the most up to date feeds. This becomes particularly apparent in a crisis situation or event where the people closest to an unfolding incident(that are unknown) could arguably be the best source of information. For the above mentioned reasons we need to rely on people with unknown trustworthiness. \\*\\*
In this dissertation therefore we focus on the problem of understanding whether or not sources with unknown trustworthiness should be relied upon. Our approach to this is to investigate and seek to identify characteristics of trustworthy and untrustworthy tweets. Using these we aim to learn patterns in the characteristics which will help us explore the possibility of developing a trust model to evaluate the trustworthiness of newly published information and sources online. 
\end{abstract}  